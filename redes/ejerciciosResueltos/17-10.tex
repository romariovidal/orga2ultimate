\documentclass[a4paper,10pt]{article}

\usepackage[spanish, activeacute]{babel}
\usepackage{a4wide}
\usepackage{enumerate}
\usepackage[utf8]{inputenc}
%\usepackage{graphicx}
\usepackage{multicol}
\usepackage{multirow}
\usepackage{latexsym}
\usepackage[dvips]{graphicx}
\usepackage{color}
\usepackage{colortbl}



\addtolength\topmargin{-1cm}
\setlength\voffset{0cm}
\setlength\headheight{0cm}
\setlength\headsep{0cm}
\addtolength\textheight{4cm}

\begin{document}

\section*{Dudas generales}
\begin{itemize}
	\item En el ejercicio 17 hay un broadcast a la direcci'on 255.255.255.255.  Nosotros habíamos entendido que las direcciones de broadcast están definidas basándose en la dirección de red, o sea en este caso 200.32.127.255.  ¿Estamos confundidos nosotros, o es una ``licencia poética'' de la práctica?
\end{itemize}

\section*{Ejercicio 2 (subnetting)}
Este va completo entero para validarlo.

Un ISP tiene la siguiente dirección: 
157.92.26.0/23

4 clientes:

\begin{itemize}
	\item a 240 hosts
	\item b 96 hosts
	\item c 16 hosts
	\item d 6 hosts
\end{itemize}

Cada cliente tiene las IPs que pidió más la dirección del enlace que se conecta con sus máquinas, más la dirección de la red, más la dirección de broadcast.


Red a:
\begin{itemize}
	\item Dirección de Red: 192.168.26.0 
	\item Máscara: 255.255.255.0 es decir /24 
	\item Broadcast: 192.168.26.255 
	\item IP del enlace: 192.168.26.254
	\item Rango de IPs: 192.168.26.1 a 192.168.26.253 Son 253 ips, desperdicio: 13
\end{itemize}


Red b:
\begin{itemize}
	\item Bits: x.x.x.1000 0000
	\item Dirección de red: 192.168.27.0 
	\item Máscara: 255.255.255.128 es decir /25
	\item Broadcast: 192.168.27.127
	\item IP del enlace: 192.168.27.126
	\item Rango de IPs: 192.168.27.1 a 192.168.27.125 Son 125 ips, desperdicio: 29
\end{itemize}


Red c:
\begin{itemize}
	\item Bits: x.x.1100 0000
	\item Dirección de Red: 192.168.27.192
	\item Máscara: 255.255.255.224 es decir /27
	\item Broadcast: 192.168.27.223
	\item IP del enlace: 192.168.27.222
	\item Rango de IPs: 192.168.27.193 a 192.168.27.221 Son 29 ips, desperdicio:13
\end{itemize}


Red d:
\begin{itemize}
	\item Bits: x.x.x.1110 0000
	\item Dirección de red: 192.168.27.224
	\item Máscara: 255.255.255.240 es decir /28
	\item Broadcast: 192.168.27.239
	\item IP del enlace: 192.168.27.238
	\item Rango de IPs: 192.168.27.225 a 192.168.27.237 Son 13 ips, desperdicio:7
\end{itemize}

\section*{Ejercicio 20}
¿Alcanza con hacer un túnel (X.25 encapsulado en IP) o para poder simular el circuito virtual asegurar que los paquetes lleguen, etc. hay que agregar algo más que encapsule IP (o entre IP y X.25)?

\section*{Ejercicio 24}
¿Qué representan los rayos amarillos??/? ¿Podría haber una subnet que abarque más de una VLAN?

\section*{Ejercicio 26}
¿``DHCP DISCOVER Fragmentado''? ¿Qué es eso? (no lo encontramos en el libro, ni se dió en clase ni lo encontramos)

\section*{Ejercicio 27}
El último punto nos genera dos dudas: una es cómo sabe IP el MTU del L2 subyacente y la más importante es qué pasa si el protocolo superior le pide a IP envíar un dato mayor al máximo largo de paquete IP. ¿IP envía varios paquetes independientes o envía un paquete fragmentado?

\section*{Ejercicio 28}
No se nos ocurre nada que a todos nos satisfaga, o al menos que nos convenza a todos.  Puede pasar que haya dos servidores de DHCP que estén otorgando las mismas IPs, puede que haya dos máquinas a las que se les estén configurando las IPs a mano o alguna otra que no se nos ocurre.

\section*{Ejercicio 29}
Dado que la red es 131.108.1.128/25 en principio creemos que la dirección de broadcast es la IV: 131.198.1.255.  Sin embargo, esta dirección IP es una dirección pública (podría pertenecer por ejemplo a un host de la red 131.108.0.0/16) por lo cual no estamos seguros de que sea una dirección broadcast válida.

\section*{Ejercicio 33}
No estamos seguros, o al menos no de una fuente ``oficial``, del funcionamiento y uso de los tres últimos campos de la tabla.

\section*{Ejercicio 53}
¿Cómo podríamos saber la dir IP de la máquina donde se corre el comando?  Podríamos estimar a qué subred pertenece, pero no la dirección IP exacta.
Por otro lado: ¿qué tipo de interfaz es fxp0? ¿Qué indica el ``permanent'' que aparece en una entrada?

\section*{Ejercicio 59}
La longitud del segundo fragmento de la tabla (100 Bytes) no es múltiplo de 8.  ¿Esto es inválido y el ejercicio apunta a que notemos eso, o es un detalle a ignorar a los efectos del ejercicio?

\section*{Ejercicio 68}
No se entiende bien la pregunta en lo que respecta a ``dos posibles escenarios en que al menos un fragmento llegue a destino''.  Sólo se nos ocurre ``que lleguen los dos'' y ``que llegue uno s'olo''.  ¿Esta es la idea?

\end{document}

