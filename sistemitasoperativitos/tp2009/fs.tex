Linux est'a dise�ado para soportar distintos t'ipos de dispositivos
f'isicos. Incluso para un tipo espec'ifico, como un disco duro,
existen diferentes interfaces de manejo entre un fabricante de
hardware y otro. Adem'as de los dispositivos f'isicos, Linux posee
soporte para sistemas de archivos l'ogicos que lo hace interoperante
entre distintos sistemas operativos, con los siguientes prop'ositos:

\begin{itemize}

\item M'ultiples dispositivos f'isicos de hardware

\item M'ultiples sistemas de archivos l'ogicos

\item M'ultiples tipos de archivos ejecutables

\item Homogeneidad, es decir, una interfaz com'un entre los sistemas de
archivos l'ogicos y el hardware

\item Rendimiento

\item Seguridad de datos, perder o corromper datos

\item Seguridad de acceso, restricci'on a los archivos, quotas,
permisos, etc.

\end{itemize}

Tambi'en se usan dos interfaces para el manejo de filesystems. Una
interfaz para llamadas de usuarios y otra para los subsistemas del
kernel. La interfaz para los usuarios maneja archivos y directorios.
Las operaciones en archivos incluyen $open, read, close, write,
seek$, que est'an definidas en el estandar $POSIX$ y en los
directorios $readdir, creat, unlink, chmod, stat$ tambi'en definidas
en $POSIX$.

Pero la interface de los subsistemas del Kernel es bastante m'as
interesante. Esta posee estructuras de datos y funciones para
manipulaci'on directa para otros subsistemas. De hecho, existen dos
interfaces para el resto del kernel: inodos y archivos.

La interfaz de inodos posee soporte para $create()$, $lookup()$,
$link()/symlink()$, $mkdir()$, $mknod()$,etc.

La interfaz de archivos posee soporte para $open()/release()$,
$read()/write()$, $select()$, $mmap()$, $fsync()$, etc.

Volviendo con los controladores de dispositivos o drivers, Linux
posee tres tipos de drivers: $char$, $block$ y $network$. Los dos
m'as relevantes son $char$ y $block$. Los dispositivos $char$ son
aquellos que su lectura es secuencial (de a un char a la vez), como
un mouse. Los dispositivos $block$ pueden ser accedidos de cualquier
manera, pero solo puede leerse o escribirse de a bloques.

Todos los drivers soportan las operaciones mencionadas reci'en.
Adem'as, cada dispositivo puede ser accesado como si fuera un
archivo. Como el Kernel se entiende con los dispositivos de esa
manera, no es complicado agregar un nuevo dispositivo, ya que solo
es necesario implementar el c�digo espec'ifico del hardware para el
soporte de la abstracci'on de archivos.

Adem'as, el kernel posee un buffer de cach'e para mejorar el
rendimiento cuando usa dispositivos $block$. Todo acceso a un
dispositivo $block$ pasa a trav'es de un subsistema de buffer. Este
buffer aumenta considerablemente el rendimiento, minimizando las
lecturas y escrituras hacia y desde los dispositivos.

Cada dispositivo posee una cola de petici'on (request queue). Cuando
el buffer no puede responder una petici'on, se agrega una de 'estas
a esta cola y hace que la petici'on duerma (sleep) hasta que pueda
responder. Este buffer usa un espacio separado del kernel como un
thread 'unico, manejado por $kflushd$.

Existen distintos mecanismos para mover datos entre el computador
hacia los dispositivos
\begin{itemize}
\item Polling
\item DMA (acceso directo a memoria)
\item Interrupciones
\end{itemize}

En el caso de polling, el driver verifica peri'odicamente el CSR
(Control and Status Register) para ver si la petici'on ha sido
completada. Si es as'i, el dispositivo inicia la siguiente petici'on
y continua. Polling es altamente efectivo en dispositivos de
transferencia lenta, como disketteras y modems.

En el caso de DMA, el driver inicia una transferencia directa entre
la memoria del computador y el perif'erico. Esta transferencia es
concurrente con la CPU y permite que la CPU siga haciendo otras
tareas mientras se transfieren datos. Cuando la operaci'on termina,
la CPU recibe una interrupci'on.

Cuando un dispositivo desea cambiar de estado (por ejemplo,
presionar el bot'on del mouse) o reporta el final de una operaci'on,
env'ia una interrupci'on al procesador. Si las interrupciones estan
habilitadas, el procesador detiene su ejecuci'on actual y comienza a
ejecutar el c'odigo de manejo de interrupciones del kernel. El
kernel encuentra el driver correcto para invocar. Cuando una
interrupci'on es manejada, la CPU se ejecuta en un contexto
especial, es decir, todas las interrupciones quedan pendientes hasta
que la interrupci'on actual haya terminado. Debido a esto, los
manejadores de interrupciones deben ser altamente eficientes. Si
ocurre el caso que un manejador de interrupciones no puede terminar
el trabajo, reasigna el trabajo pendiente en un manejador de tipo
``bottom\-half'' , es decir, c'odigo que se ejecutar'a la pr'oxima
vez que la llamada sea completada para evitar latencias y promover
la concurrencia.

Volviendo ahora a los sistemas de archivos l'ogicos, es m'as f'acil
acceder a un dispositivo si es manejado como un archivo a que si
fuera manejado directamente en el hardware. Un filesystem l'ogico
puede ser montado en un punto de montaje dentro del filesystem
virtual. Esto significa que un bloque asociado a un dispositivo
contiene archivos y una informaci'on de estructura que permite al
filesystem l'ogico accederlo. Un dispositivo s'olo puede tener un
soporte l'ogico de archivo. Sin embargo, puede ser ``adiestrado''
para poseer soporte para un soporte l'ogico en otro tipo de
filesystem.

En pocas palabras y con una explicaci'on m'as clara, los
dispositivos por lo general se encuentran en $/dev$ y cada uno se
representa como un archivo. Es perfectamente posible reescribir su
driver para que soporte otro tipo de filesystem l'ogico.

Para el soporte de filesystems virtuales, Linux usa Inodos para
representar un archivo en un dispositivo block. El inodo es virtual
en el sentido que contiene operaciones que est'an implementadas de
distintas maneras, dependiendo del sistema l'ogico y f'isico donde
el archivo reside. El inodo es usado como un lugar de almacenamiento
para toda la informaci'on relacionada con el manejo de, por ejemplo,
abrir un archivo del disco. Guarda buffers, el largo del archivo ,
etc.

Mucha de la funcionabilidad de los filesystems virtuales radica en
los m'odulos. Esta configuraci'on permite a los usuarios compilar un
Kernel tan peque�o como sea posible, cargando solamente los modulos
que sean necesarios.


Ideal es el caso en el cual el dispositivo pudiera ocuparse y
liberarse en tiempo de ejecuci'on. Si el m'odulo fuera agreagado
activamente al kernel, se deben pasar sus par'ametros en momento de
booteo, sin considerar tampoco el gasto de memoria en mantener el
uso de un dispositivo que no est'e conectado, como por ejemplo, una
impresora.
