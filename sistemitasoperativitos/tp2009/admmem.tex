El Memory Manager o administrador de memoria permite lo siguiente:

\begin{itemize}

\item Brindar gran espacio de direcciones : Los programas pueden requerir m'as
memoria que la que f'isicamente existe

\item Protecci'on : La memoria asignada a un proceso es privada para tal

\item Mapas de Memoria : Se puede mapear un archivo dentro de un area de
memoria virtual y acceder al mismo como si fuera memoria
convencional

\item Acceso Limpio a la Memoria F'isica : el MM asegura que los procesos
puedan usar transparentemente todos los recursos de la m'aquina,
asegurando adem'as un rendimiento aceptable

\item Memoria Compartida : Permite que los procesos puedan compartir
trozos de la memoria asignada.

\end{itemize}

El Administrador de memoria posee dos interfaces : una, con llamada
a sistema que es usada por los procesos de usuario (User Space) y la
otra, es utilizada por los otros subsistemas (Kernel Space).

Algunas de las llamadas de usuario incluyen $malloc()$ y $free()$.
Las llamadas de kernel comprenden $kmalloc()$ y $kfree()$.

Como Linux soporta multiples arquitecturas, es necesario entonces
que existan rutinas espec'ificas para abstraer los detalles del uso
del hardware en una sola interface.

El Memory Manager usa al administrador de memoria de hardware para
mapear direcciones virtuales a direcciones f'isicas. Gracias a esto,
los procesos no est'an concientes de cuanta memoria f'isica est'a
asociada a una direcci'on virtual. Esto permite al MM poder mover
trozos de memoria virtual dentro de la memoria f'isica. Adem'as,
permite que dos procesos puedan compartir trozos de memoria f'isica
si las regiones de memoria virtual asignadas son mapeadas en la
misma direcci'on f'isica.

Otro concepto importante es el Swapping o intercambio, que se
describe como intercambiar (swap) memoria ocupada por procesos a un
archivo. Esto le permite al kernel poder ejecutar mayor cantidad de
procesos que usen mayor cantidad de memoria que la f'isica
existente. El MM posee un m'odulo llamado $kswapd$, que sirve para
ejecutar la tarea de intercambiar zonas de memoria en archivos y
viceversa. Adem'as, chequea periodicamente si no existen direcciones
f'isicas mapeadas recientemente. Estas direcciones son vaciadas de
la memoria f'isica. Cabe destacar que el MM busca minimizar la
cantidad de actividad de disco necesaria relacionada con este
intercambio(minimizar el trashing).

El hardware del administrador de memoria, detecta cuando un proceso
de usuario tiene acceso a una porcion de memoria no mapeada en una
direcci�n virtual y notifica al kernel de esta falla. Existen dos
alternativas para solucionar esto: o la p'agina de memoria es
volcada a un archivo o viceversa; o el proceso est� haciendo
referencia a una zona de memoria no permitida.

Si el administrador de memoria detecta un acceso no permitido (a la
memoria), notifica al proceso con una se�al. Si el proceso no sabe
como manejar esta se�al, entonces, es finalizado.
