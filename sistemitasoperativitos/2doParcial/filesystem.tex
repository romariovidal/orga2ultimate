\subsection{Introducci'on}

El sistema de archivos pose'e 2 partes:
\begin{itemize}
	\item una colecci'on de \textbf{archivos}: cada uno con datos
	\item una \textbf{estructura de directorios}: para organizarlos
\end{itemize}
algunos adem'as tienen particiones.

\subsection{Archivo}
Es una colecci'on de informaci'on relacionada. Es lo m'as peque���o que podemos
escribir en memoria secundaria.

Representan programas o datos, pero en realidad son tiras de biy, bytes o
registros que son interpretados como una u otra cosa (estructura del archivo).

\subsubsection{Atributos} 
\begin{itemize}
  \item \textbf{nombre}: 'unica informaci'on entendible por humanos as'i como
est'a
  \item \textbf{tipo}
  \item \textbf{ubicaci'on}: puntero a dispositivo y a una posici'on en 'el
  \item \textbf{tama\~no}
  \item \textbf{protecci'on}
  \item \textbf{hora, fecha e identificaci'on del usuario}: datos 'utiles para
protecci'on, seguridad y control del uso.
\end{itemize}

La informaci'on de todos los archivos se guarda la info de todos los archivos.

\subsubsection{Operaciones}
Operaciones que realiza un sistema con archivos: 
\newpage
\begin{itemize}
  \item \textbf{Crear un archivo} Se debe encontrar el lugar donde estar'a y
luego agregarlo al directorio.
  \item \textbf{Escribir un archivo} Se hace una llamada al sistema con el
nombre del archivo y lo que hay que escribir. Este lo busca y lo escribe.
Interviene un puntero a escritura, para saber por donde va escribiendo.
  \item \textbf{Leer un archivo} Se hace una llamada con el nombre y el lugar
donde debe colocarse el siguiente bloque del archivo. Tambi'en tiene un puntero
a la pr'oxima secci'on a leer. Se suele usar un 'unico puntero para las 2 op
(r/w).
  \item \textbf{Reubicarse dentro de un archivo} No implica E/S. Es como una
b'usqueda.
  \item \textbf{Eliminar un archivo} se libera el espacio y se borra la entrada
del directorio.
  \item \textbf{Truncar un archivo} Se usa para borrar un archivo pero no los
atributos (poner la long en 0)
\end{itemize}

Con este set de instrucciones se pueden hacer todas las operaciones.

Para evitar tener que buscar varias veces un archivo abierto en el directorio,
los SO tienen una \textbf{Tabla de Archivos Abiertos}.

En sistemas complejos, ej multiusuario, hay adem'as una Tabla de Archivos
Abiertos por cada proceso que figura en la Tabla de Procesos. En estos casos
hay un contador de aperturas para cada archivo.

Los archivos pueden tener extensiones para determinar de que tipo son, o tener
atributos extras. 

\subsubsection{M'etodos de acceso}
Secuencial: lineal, en orden, con un puntero que va avanzando.

Directo: Accede a bloques, ya que los discos permiten acceso aleatorio a
cualquier bloque. Sin orden. Las bases de datos usan este tipo.

Mediante 'indices: con una tabla, como si fuera \textbf{realmente un 'indice}.
Puede haber 2 niveles de 'indices.

\subsection{Estructura de directorios}
Los discos se organizan en particiones, que forman estructuras l'ogicas (de
mayor tama~no o menor que un disco) que el usuario trata como un dispositivo.
Cada partici'on tiene un \textbf{directorio} en el que se registran los datos de
todos los archivos de la misma. El directorio nos debe prestar las siguientes
operaciones:
\begin{itemize}
  \item \textbf{Buscar un archivo} por nombre o por patr'on
  \item \textbf{Crear un archivo}
  \item \textbf{Eliminar un archivo}
  \item \textbf{Listar un directorio}
  \item \textbf{Cambiar el nombre de un archivo}
  \item \textbf{Recorrer el sistema de archivos}
\end{itemize}

\subsubsection{Niveles de un directorio}
\subsubsection*{Un nivel}
La m'as sencillas. No es escalable. M'as de un usuario podr'ia querer crear un
archivo llamado ``prueba'' o ``puto'' y se violar'ia la regla de nombres
'unicos.

\subsubsection*{Dos niveles}
Aca se puede crear un directorio por usuario y listo. As'i, cada usuario tiene
su directorio de archivos de usuario (UFD user file directory). Hay un
directorio de archivos maestro (MFD master file...) que sabe donde est'an los
UFD de cada usuario.

Se debe agregar un directorio de usuario al cual se debe poder acceder desde el
MFD.

Para que un usuario pueda acceder a un archivo de otro usuario debe proveer el
nombre del due~no del mismo y el nombre del archivo, formando un camino (path)
entre la ra'iz (MFD) y la hoja (archivo) del 'arbol que es el sistema de
archivos.

Se agrega un ``usuario especial'' que tiene todos los programas, entonces
cuando un usuario quiere correr uno, primero se busca en su directorio local y
despu'es en el del usuario especial. De esta manera se evita tener que copiar
todos los archivos del sistema a los directorios de cada usuario. (Camino de
busqueda)

\subsubsection*{'Arboles o m'as de dos niveles}
Permite crear subdirectorios. Cada entrada del directorio se define con un bit
que lo identifica como archivo (0) o como subdirectorio(1). Aparece el concepto
de directorio actual para un usuario, antes era su 'unico directorio, ahora es
alguno de sus subdirectorios. Se agrega la llamada al sistema de
\textbf{cambiar directorio (cd)}. Hay path absolutos (desde la ra'iz) o
relativos (desde el directorio actual)

\subsubsection*{Grafo ac'iclico}
Es para compartir archivos y/o directorios, no tiene estructuras c'iclicas. La
forma m'as facil de implementar esto es poner un link (puntero) al mismo
archivo. Tambi'en se puede duplicar todo para compartir, pero es un garr'on
para mantener.

Hay que tener cuidado de que si queremos contar los archivos que tenemos, no
contar 2 veces el mismo por 2 ramas de nuestro directorio.

Tambi'en hay que tener cuidado cuando se borra. Si se borra el archivo, quedan
colgando los links (punteros), ser�a costoso buscarlos todos, a menos que
guardemos una lista de links al archivo en nuestro archivo o un conteo de
referencias que nos deje borrar s'olo cuando est'a en cero.

DOS no permite grafos ac'iclicos porque es demasiado complicado.

\subsubsection*{Grafo general - General Graph, companiero de General Electric}
Nada, bardo, la busqueda mal dise~nada da facilmente un algoritmo infinito.

Es dificil saber si al agregar un link no estamos cayendo en un grafo con
ciclos.

\subsection{Protecci'on}
Contra da~nos f'isicos (confiabilidad) se logra con backups, contra acceso
indebido (protecci'on) se logra en un sistema monousuario retirando
f'isicamente los discos flexibles y guard'andolos bajo llave en un caj'on del
escritorio o en un archivero (sic).

\subsubsection{Tipos de acceso}
En los que no se permite el acceso a los archivos de otros usuario no se
necesita protecci'on. Los tipos de acceso se separan por la operaci'on que se
quiere realizar en cada uno (de los accesos):
\begin{itemize}
  \item Leer
  \item Escribir
  \item Ejecutar
  \item Anexar (append) agregar informaci'on al final del archivo.
  \item Eliminar, borrar el archivo y liberar el espacio
  \item Listar el nombre y los atributos de un archivo.
\end{itemize}

Las dem'as operaciones se pueden hacer con estas, ej copiar un archivo es leer
muchas veces y escribir un archivo.

\subsubsection{Listas y grupos de acceso}
Lo mejor es que el accedo dependa de qui'en es el que quiera acceder. Para eso
se tiene una lista de acceso que especifica el nombre de un usuario y sus
permisos. Cuando un usuario solicita eacceso a un archivo, el sistema mira la
lista de acceso asociada a ese archivo. Si el usuario est'a autorizado para el
acceso que pidi'o, entonces se lo permite, sino... no, ovbio.

El problema de estas tablas es el tama~no. Una tabla que permita el acceso a
TODOS los usuario puede ser muy grande, antes la entrada de un directorio
ten'ia tama~no fijo. Para solucionar esto hay 3 categorioas en relaci'on con
cada archivo:
\begin{itemize}
  \item Propietario
  \item Grupo (cjto de usuarios que comparten el archivo o que necesitan acceso
similar)
  \item Universo (los dem'as, el resto, la plebe, la gilada, etc)
\end{itemize}
Si usamos un bit para read R, uno para write W y otro para execute X, nos queda
que con 3 campos de 3 bits tenemos todas las combinaciones de RWX con los
grupos anteriores.
La opci'on es tener una Lista de Control de Acceso por Archivo, con la anterior
clasificaci'on es bastante f'acil y peque.

Falta lo de LCU, que dice a que archivos puede acceder cada usuario. PODEMOS
COMPLETAR ESTO CON EL APUNTE MA�ANA EN LA FACU.

Otra opci'on es ponerle contrase~na a los archivos, pero es demasiado pesado
para el usuario recordar muchas, con lo cual es una pass a ``todo o nada''.

\subsection{Estructura}
Las transferencias entre la memoria se realizan en bloques. Cada bloque ocupa
uno o m'as sectores. Dependiendo el disco el sector var'ia de tama~no entre 32
y 4096 bytes, en gral son de 512

ME QUEDAN UNAS 6 HOJAS, PERO YA NO ENTIENDO NADA.
