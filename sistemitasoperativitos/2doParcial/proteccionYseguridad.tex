Los mecanismos de protecci'on proveen el acceso controlado y restringido de los recursos a los procesos que obtienen la autorizaci'on adecuada.
El sistema de seguridad previene el acceso sin autorizaci'on al sistema y sus recursos.

\subsection{Protecci'on}

El rol de la protecci'on en un sistema es proveer los ``mecanismos'' encargados de hacer cumplir las ``pol'iticas'' de utilizaci'on de los recursos.

Las pol'iticas de utilizaci'on de los recursos pueden cambiar, por esta raz'on el sistema operativo lo que tiene que proveer son los mecanismos de protecci'on al dise�ador de las aplicaciones para poder manipularlas de forma din'amica.

Es muy importante separar las pol'iticas de los mecanismos, las pol'iticas dicen que se tiene que hacer y los mecanismos como. Con robustos mecanismos uno puede llegar a tener que cambiar solo algunos parametros en caso de que cambien las pol'iticas de uso.

\subsection{Dominio}

Un sistema es una colecci'on de objetos y procesos. Un objeto puede ser software o hardware y se identifica univocamente. Estos son esencialmente un tipo abstracto de dato. A estos objetos se le pueden asignar operaciones que dependen del objeto (leer, escribir, ejecutar, etc).

Entonces, los procesos pueden realizar estas operaciones sobre los objetos siempre y cuando tengan la autorizaci'on adecuada. Adem'as, los procesos solo deber'ian tener autorizaci'on necesaria para realizar su tarea (objetos y operaciones).

Los procesos ejecutan sobre un dominio. Los dominios son los que tienen estos objetos con sus respectivas operaciones. Un dominio es una colecci'on de permisos de acceso, un par objeto y conjunto de operaciones permitidas ($<$archivo 1,$[$leer, ejecutar$]>$). Estos dominios pueden compartir accesos, no necesariamente son disjuntos.

Esta asociaci'on entre procesos y dominios no puede ser fija, ya que al cambiar alguna pol'itica un proceso podr'ia necesitar mas autorizaci'on o perder algunos accesos. Por esto, se deben proveer mecanismos de modificaci'on o resignaci'on de dominios.

Formas de realizar los dominios:
\begin{itemize}
	\item Los usuarios pueden ser dominios, los procesos acceden a los objetos con la identidad del usuario. Un switch se hace con un loggeo de un distinto usuario.
	\item Los procesos pueden ser dominios, cada proceso tiene su conjunto de accesos y un switch se hace con un mensaje a otro proceso.
	\item Los procedimientos pueden ser dominios, los objetos que pueden ser accedidos corresponden a las variables locales del procedimiento. Un switch es la llamada de un procedimiento.
\end{itemize}

\textbf{Ejemplos}

El modelo dual est'andar (monitor-usuario) cuando se ejecuta en modo monitor se pueden ejecutar las rutinas del sistema mientras que en modo usuario solo se tienen instrucciones no privilegiadas.

\textbf{Ejemplos - Unix}

Unix utiliza la asociaci'on de dominios a usuarios. El cambio de dominio se realiza con un cambio temporal de usuario. Unix utiliza el modelo de file system, a cada archivo se le asocia un due�o y un bit de dominio. Cuando se ejecuta un archivo cuyo due�o es otro usuario y tiene el bit de dominio apagado el proceso se inicia con el usuario que lo ejecuta y sus permisos, sin embargo, si el bit esta encendido se le asigna el usuario due�o al procesos y sus permisos correspondientes hasta la finalizaci'on del proceso.
El problema con este mecanismo es que si un usuario logra crear un archivo con due�o a "root" u el bit de dominio encendido puede tener acceso sin restricciones al sistema.
Un m'etodo de mitigaci'on de este problema es asignar los permisos por directorio y realizar el cambio de usuario solo al due�o del directorio. Esto seria menos flexible.
Algo incluso mas restrictivo es no permitir el cambio de usuario y realizar una estrategia de proceso "daemon" al cual invocar para acceder a instrucciones privilegiadas.

\textbf{Multics}

Multics organiza los dominios como anillos ger'arquicos, 0 al 7 con sus dominios D0 a D7. si $i < j$ entonces Dj es un subconjunto de Di.

Multics usa direcci'on de memoria segmentada, cada segmento es un archivo asociado a un anillo. El descriptor incluye el numero del dominio y 3 bits de control, escritura, lectura y ejecuci'on. Cada proceso tiene asociado un numero de dominio y este puede acceder a los segmentos mayor o igual a este numero y respetando los bit de control.

Para el intercambio de dominio se tienen en cuenta los siguientes cambios en los descriptores de segmento:
\begin{itemize}
	\item Accesos: Dos enteros, $b1 <= b2$
	\item Limite: Un entero $b3 > b2$
	\item Gates: Puntos de entrada que puede llamar el segmento.
\end{itemize}

Si un proceso del anillo i llama a un procedimiento (segmento) con accesos b1, b2 tal que $b1 <= i <= b2$ el proceso continua con nivel i. Sino:

Si $i < b1$ el procedimiento se ejecuta, sin embargo, los datos deben ser copiados a un segmento con acceso de i.

Si $b2 < i$ la llamada es permitida solo si $b3 <= i$ y se utiliza un gate.

El principal defecto de esta t'ecnica es que no se restringen los permisos solo a los necesarios.

\subsection{Matriz de accesos}

Los permisos pueden verse como una matriz abstracta donde las filas son los dominios y las columnas son los objetos (ver figura 19.3). Los elementos de esta matriz son los conjuntos de operaciones permitidas. Un proceso que se ejecuta en el dominio i solo tiene los permisos que figuran en la fila i. De esta forma podemos implementar las pol'iticas de acceso con esta matriz.
Para realizar un cambio de dominio se utiliza una estrategia de colocar a los dominios como objetos tambi'en (como columnas) y utilizar una operaci'on "switch". Si switch esta en (i,j) entonces el dominio de fila i puede cambiar al dominio de la columna j.
Para permitir el cambio de esta matriz de forma din'amica tenemos que agregar los siguentes operaciones: copiar, due�o y control.
La posibilidad de copiar se denota como un (*) al lado de la operaci'on de acceso. hay 3 tipos de permisos de copia: copia id'entica (con el * tambi'en), copia limitada (solo la operaci'on de acceso sin el *) o transferencia (mover el permiso). Estas operaciones son siempre en el mismo objeto, es decir, en la misma columna.
Un proceso con el permiso de due�o (denotado owner) puede crear nuevas operaciones de acceso o eliminar en cualquier campo de la misma columna.
Estos 2 permisos nos proveen mecanismos para limitar la propagaci'on de accesos, sin embargo, no la propagaci'on de la informaci'on de los objetos fuera del entorno de ejecuci'on. Este problema se llama problema de confinamiento y en general no es resuelto.
La operaci'on de control es solo aplicable a objetos de tipo dominio (dominio como columna). Control en el elemento (i,j) nos dice que el dominio i puede modificar los accesos de la fila dominio j.
Recordar que la importancia de estas 3 operaciones es permitir la modificaci'on dinamica de la matriz de accesos.

\subsection{Implementaciones de la matriz de accesos}

\textbf{Tabla global}: \\Conjunto de tuplas $<$dominio, objeto, conjunto de permisos$>$.

\textbf{Lista de accesos por Objeto}: Por objeto se tiene una lista de duplas <dominio, conjunto de permisos> (dominios con conjunto de permisos no vac'ios).

\textbf{Lista de capacidad por dominios}: 'idem anterior pero usando las filas en lugar de las columnas. La lista de capacidades es en si un objeto protegido y mantenido por el sistema operativo. Estos objetos se diferencian de los datos por tener un tag que los identifica o utilizando una divisi'on de la memoria entre datos y capacidades (memoria segmentada).

\textbf{Candado-llave}: Este utiliza lista de accesos (con candados) y lista de capacidades (con llaves). Un proceso en un determinado dominio debe tener la llave para abrir el candado si quiere acceder a un objeto.

\subsection{Comparaci'on entre implementaciones}

Lista de accesos corresponde directamente a las necesidades del usuario asignando los permisos por objeto. Sin embargo, hay que checkear los accesos al procesar en todo momento. Una lista larga de accesos puede demorar esta tarea.

Lista de capacidades se corresponde mas a los procesos, es mas f'acil identificar a la hora de ejecuci'on. Sin embargo, el quitar permisos es complicado a la hora de redefinir permisos.

Candado -llave, al ser un mix de las 2 puede ser efectiva y flexible dependiendo del largo de las llaves.

Lo mas usado es un mix de los 2 primeros (ej. tener lista de accesos y en cada ejecuci'on ir generando una lista de capacidades que se elimina al finalizar el proceso).

\subsection{Revocar permisos de acceso}

\begin{itemize}
	\item inmediatamente vs demorado
	\item selectivo vs general
	\item parcial vs general
	\item temporal vs permanente
\end{itemize}

Si se utiliza lista de accesos es sencillo y se puede realizar de la forma que se desee.
Con lista de capacidades se plantean varias estrategias:

\textbf{Readquisici'on}: La lista se borra peri'odicamente de los dominios y hay que actualizarla.

\textbf{Punteros}: Una lista de punteros se mantiene en el objeto a las capacidades y se revoca utilizando estos punteros (es costoso, se usa en multics).

\textbf{Indirecci'on}: Las capacidades apuntan a una entrada en una tabla global de puntero a los objetos, cuando se quita el objeto de la tabla se pierde el acceso por fallo en la indirecci'on (puntero no valido). No permite m'etodo selectivo, se revoca a todos los dominios.

\textbf{Llaves}: Una llave maestra se le asigna a los accesos del objeto, cuando se crea una capacidad en un dominio se copia esta llave. El cambio de llave revoca de forma general el acceso a todos los dominios. Este m'etodo tambi'en puede implementarse con varias llaves para tener mas flexibilidad y proveer una revocaci'on mas selectiva.

\subsection{Sistemas basados en capacidades}

Hydra (descripci'on no relevante, ir al libro)

Cambridge CAP (descripci'on no relevante, ir al libro)

\subsection{Protecci'on basada en lenguaje}

La protecci'on de los recursos puede variar por aplicaci'on o cambios de pol'iticas, por lo que se comenz'o a considerar un problema no solo de los dise�adores de sistemas operativos, sino tambi'en, de los dise�adores de aplicaciones. Por esto, los lenguages de programaci'on se comenzaron a utilizar para brindar protecci'on. Cuando la ``declaraci'on'' de la protecci'on se adjunta a la declaraci'on de los tipos de datos se permite especificar los requerimientos de protecci'on al dise�ador del subsistema. Ventajas:
\begin{itemize}
	\item Las protecciones son simplemente declaradas, en lugar de llamar a subrutinas del sistema operativo.
	\item La protecci'on se independiza de las facilidades provistas por el sistema operativo.
	\item Los medios de ejecuci'on no se proveen por parte del dise�ador del subsistema.
	\item Una notaci'on declarativa es mas natural ya que los privilegios de acceso est'an relacionados a los conceptos ling�isticos de los tipos de dato.
\end{itemize}
(l'ease tipos de datos como los objetos a restringir, al menos es lo que yo entend'i).

Todos los metodos de protecci'on dependen en cierto grado de la implementaci'on de la protecci'on del sistema operativo. La principal distinci'on entre tener una buena implementaci'on en el sistema operativo es la seguridad de la protecci'on brindada por el kernel, ya que los mecanismos se basan en como opera el sistema (un compilador puede asegurar que no hay huecos en el soft pero no que el archivo en ejecuci'on no mute estas protecciones).

\textbf{Seguridad}: Mas relevante en el kernel, a nivel compilaci'on se basa en la correcta traducci'on, mecanismos de protecci'on de segmentos para el manejo del almacenamiento y la seguridad de los archivos de los que se carga el programa.

\textbf{Flexibilidad}: Hay muchos limites a nivel kernel para implementar pol'iticas especificas de usuarios y es muy costoso la modificaci'on del kernel. En cambio, a nivel lenguaje la modificaci'on afecta menos al sistema general.

\textbf{Eficiencia}: La ejecuci'on de protecci'on por hardware es la mas eficiente. Sin embargo, soporte del software en necesario y un compilador inteligente puede evitar sobreprocesamiento por parte del kernel.

\subsection{Resumen}
La especificaci'on de la protecci'on en un lenguaje provee una descripci'on de las pol'iticas en un alto nivel. la implementaci'on provee protecci'on no brindada por el hardware y se pueden interpretar llamadas a rutinas de protecci'on en el kernel.

%
%
%
%
%

\subsection{Seguridad}

Seguridad, a diferencia de protecci'on, no solo requiere una protecci'on adecuada del sistema, sino que tambi'en se deben considerar factores externos. Se debe protejer de uso sin autorizaci'on, destrucci'on o alteraci'on intencional o inconsistencias accidentales.

\subsection{El problema de la seguridad}

Un sistema se dice seguro si sus recursos se acceden y utilizan como corresponde bajo cualquier circunstancia.
Violaciones de seguridad pueden ser maliciosas o accidentales. las maliciosas se categorizan como lectura, modificaci'on o destrucci'on de informaci'on (desautorizada).

No es posible una protecci'on absoluta, solo aumentar la dificultad de acceso lo mas posible.
Para proteger el sistema se deben tomas 2 medidas de seguridad f'isica (el lugar donde esta el hardware) y humana (que una persona de accesos a otro sin autorizaci'on). La seguridad que vamos a estudiar es a nivel operativo, la cual queda insignificante ante una falla de los dos aspectos mencionados.

\subsection{Autentificaci'on}

La habilidad de identificar un usuario. Hay 3 aspectos: posesi'on (llave o tarjeta), conocimiento (identificador o contrase�a) y atributos (huella digital, firma).
Password (contrase�a)

Es muy flexible, se puede asignar a un usuario o a recursos espec'ificos. Es muy com'un y de concepto sencillo.
Las vulneravilidades pueden ser varias. Un password sencillo puede ser adivinado o encontrado por fuerza bruta. Un password complicado (may'usculas, min'usculas, n'umeros y s'imbolos) y largo puede ser escrito por el usuario. Visualizaci'on del password, ya sea mirando el teclado o un m'etodo esp'ia por medio de una red. Se pueden proponer cambios de password peri'odicos, incluso sin dejar repetir los N 'ultimos como metodos de seguridad. Si los password se comparten puede ser complicado identificar el origen de la infracci'on. Los password pueden ser elegidos por los usuarios o generados autom'aticamente (mas dif'icil de recordar).

\subsection{Password encriptados}
Se utiliza una funci'on f'acil de ejecutar pero dif'icil de invertir. La tabla con los password no necesita ser protegida, sin embargo, teniendo la funci'on se puede realizar fuerza bruta para comparar los resultados con la tabla.

\textbf{Password de una vez}: 
Este m'etodo utiliza un secreto y una semilla. El sistema realiza una pregunta al usuario utilizando la semilla y este contesta utilizando el secreto. El sistema conoce tambi'en el secreto y puede calcular el resultado para compararlo con el que dio el usuario. Este m'etodo es muy efectivo para evitar capturas de password, ya que todas las veces se genera un nuevo password al cambiar la semilla(ej: secreto: fn cuadrado, semilla: 2, respuesta: 4). Una variante puede ser una lista de password que se utilizan en orden.

\textbf{Utilizaci'on de programas}: 
Un programa escrito por una persona puede ser utilizado por otra.

\textbf{Caballo de Troya}:
Un programa que trata de ejecutarse en un dominio con mayor permiso causando quiebres de seguridad. Una variante es aparentar una ventana de logueo haciendo que el usuario ingrese su contrase�a.

\textbf{Puerta trampa (trap door)}:
 Huecos dejados de forma intencionada por el productor de un software ara utilizarlos mas tarde. Una puerta de entrada mus inteligente incluso puede estar en un compilador.

\textbf{Utilizaci'on del sistema}:
Los sistemas operativos proveen la posibilidad de que un proceso ejecute otro proceso, esto posibilita dos tipos de mala utilizaci'on.

\textbf{Gusanos}:
Los gusanos utilizan estos metodos para reproducirse copiandose a si mismo tomando los recursos del sistema y no permitiendo el uso de ellos por el usuario. Estos son muy peligrosos ya que pueden reproducirse incluso a travez de la red.

\textbf{Virus}:
Un virus es un fragmento de codigo que se agrega a un programa con la finalidad de da�ar o destruir informaci'on. La diferencia principal con el gusano es que el virus no es una entidad autosuficiente. Los principales afectados son los usuarios de microcomputadoras. Los antivirus son una forma efectiva para controlarlos, estos buscan patrones de ejecuci'on utilizados por los virus en los programas y quitan esos fragmentos de codigo.

Un metodo mas efectivo es la prevenci'on o utilizaci'on de mecanismos para verificar la consistencia de los programas (ej. contrar el tama�o de los ejecutables y notificar los cambios)

\subsubsection{Utilizaci'on de monitoreo}
Pueden verse 2 aspectos, revisar patrones sospechosos (fallas repetidas de password) o llevar una bitacora (log) de los sucesos.

Otro metodo es realizar revisiones frecuentes y reparar automaticamente o avisar sobre aspectos irregulares:
\begin{enumerate}
	\item Password cortos
	\item Seteo de id sin autorizaci'on a un programa
	\item Programas nuevos sin autorizaci'on
	\item Procesos largos que no se esperaban
	\item Protecciones inadecuadas en directorios de usuario y sistema
	\item Protecciones inadecuadas en archivos privilegiados (lista de password, kernel)
	\item Entradas peligrosas, como troyanos
	\item Cambios en programas del sistema encontrados por checkeo de suma (checksum)
\end{enumerate}

Otro problema es la conexion remota a un sistema. Una protecci'on de esto es la utilizacion de un firewall, es un interlocutor entre 2 sistemas filtrando la informaci'on que intercambian.

\subsection{Encriptaci'on}
TODO...