\section{Sem'aforos}
Un sem'aforo es una variable especial protegida (o tipo abstracto de datos) que constituye el m'etodo cl'asico para restringir o permitir el acceso a recursos compartidos (por ejemplo, un recurso de almacenamiento del sistema o variables del código fuente) en un entorno de multiprocesamiento (en el que se ejecutar'an varios procesos concurrentemente). Fueron inventados por Edsger Dijkstra y se usaron por primera vez en el sistema operativo THEOS.
En este simulador permitimos armar los sem'aforos de Dijkstra 

