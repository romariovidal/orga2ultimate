\section{Algoritmo del Banquero}
El Algoritmo del banquero, en sistemas operativos es una forma de evitar el \emph{deadlock}, propuesta por primera vez por Edsger Dijkstra. Es un acercamiento te'orico para evitar los \emph{deadlocks} en la planificaci'on de recursos. Requiere conocer con anticipaci'on los recursos que ser'an utilizados por todos los procesos. Esto 'ultimo generalmente no puede ser satisfecho en la pr'actica.
El algoritmo mantiene al sistema en un estado seguro. Un sistema se encuentra en un estado seguro si existe un orden en que pueden concederse las peticiones de recursos a todos los procesos, previniendo el \emph{deadlock}.
Los procesos piden recursos, y son complacidos siempre y cuando el sistema se mantenga en un estado seguro despu'es de la concesi'on. De lo contrario, el proceso es suspendido hasta que otro proceso libere recursos suficientes.
En t'erminos m'as formales, un sistema se encuentra en un estado seguro si existe una secuencia segura. Una secuencia segura es una sucesi'on de procesos, $< P_1,\ldots, P_n >$ , donde para un proceso $P_i$, el pedido de recursos puede ser satisfecho con los recursos disponibles sumados los recursos que están siendo utilizados por $P_j$, donde $j < i$. Si no hay suficientes recursos para el proceso $P_i$, debe esperar hasta que alg'un proceso $P_j$ termine su ejecuci'on y libere sus recursos. Reci'en entonces podr'a $P_i$ tomar los recursos necesarios, utilizarlos y terminar su ejecuci'on. Al suceder esto, el proceso $P_{i+1}$ puede tomar los recursos que necesite, y as'i sucesivamente. Si una secuencia de este tipo no existe, el sistema se dice que está en un estado inseguro, aunque esto no implica que esté bloqueado.
Así, el uso de este tipo de algoritmo permite impedir el \emph{deadlock}, pero supone una serie de restricciones:
\begin{itemize}
 \item Se debe conocer la máxima demanda de recursos por anticipado.
 \item Los procesos deben ser independientes, es decir que puedan ser ejecutados en cualquier orden. Por lo tanto su ejecución no debe estar forzada por condiciones de sincronización.
 \item Debe haber un número fijo de recursos a utilizar y un número fijo de procesos.
 \item Los procesos no pueden finalizar mientras retengan recursos.
\end{itemize}
En este trabajo realizamos un simulador para el Algoritmo del Banquero. La idea principal de este simulador es mostrar paso a paso la ejecuci'on del algoritmo para una instancia que el usuario provea, asi el usuario podra saber si se trata de un sistema que se encuentra en estado seguro o inseguro.